\def\VCDate{2018/03/15}\def\VCVersion{(Current)}
\documentclass{ffslides}
\ffpage{25}{\numexpr 16/9}
\usepackage{ProofPower}
\usepackage{amsmath,amssymb,arydshln}
\usepackage{framed,multicol,graphicx}
\usepackage{showframe}
\begin{document}
\obeyspaces

\normalpage{CS113 LAB 4 - Truth Table}{
\begin{GFT}{SML}
\+fun ttgen\_ex((a:'a,b:'a),[ ]) = [ ] | \\
\+    ttgen\_ex((a:'a,b:'a), q::qs : 'a list list) = \\
\+        [q @ [a]] @ [q @ [b]] @ ttgen\_ex((a,b), qs) : 'a list list;\\
\+fun ttgen(0, (\_, \_)) = [[ ]] : 'a list list |\\
\+    ttgen(1, (a:'a,b:'a)) = [[a],[b]] |\\
\+    ttgen(d, (a:'a,b:'a)) = ttgen\_ex((a,b), ttgen(d-1, (a,b)));\\
\end{GFT}
}




\normalpage{CS113 LAB 4 - SML Functions}{
\begin{GFT}{SML}
\+fun conditional [false, false] = true | conditional [false, true] = true\\
\+  | conditional [true, false] = false | conditional [true, true] = true;\\
\+fun disjunction [false, false] = false | disjunction [false, true] = true\\
\+  | disjunction [true, false] = true | disjunction [true, true] = true;\\
\+fun isValid([ ], c, p, q) = c[p, q]\\
\+  | isValid(pr::ps, c, p, q) = if pr[p, q]\\
\+    then isValid(ps, c, p, q) else true;\\
\+fun isTautology([ ]) = true\\
\+  | isTautology(x::xs) = x andalso isTautology(xs);\\
\end{GFT}
}
\btext{.60}{.15}{.35}{Algorithm to determine if an argument is valid\\
\qi{Identify the premises and the conclusion of the argument}
\qi{Contruct a truth table including premises and conclusion}
\qi{Find all rows in which premises are true}
\qi{If in each of the true rows, the conclusion is true then argument is valid}
\qii{i.e. a tautology}}
\putfig{.05}{.63}{.50}{conditional}
\putfig{.05}{.70}{.50}{disjunction}
\putfig{.05}{.78}{.80}{isValid}
\putfig{.05}{.84}{.70}{isTautology}




\normalpage{CS113 LAB 4 - Proof Power Rules}{
\begin{GFT}{SML}
\+val premise = asm\_rule;\\
\+val modus\_ponens = \PrLE{}\_elim;\\
\+val modus\_tollens = modus\_tollens\_rule;\\
\+val disjunctive\_addition = \PrLC{}\_right\_intro;\\
\+val conjunctive\_addition = \PrLB{}\_intro;\\
\+val conjunctive\_simplification = \PrLB{}\_right\_elim;\\
\+val disjunctive\_syllogism = \PrLC{}\_cancel\_rule;\\
\+val hypothetical\_syllogism = \PrLE{}\_trans\_rule;\\
\+val double\_negation = \PrLD{}\_\PrLD{}\_elim;\\
\end{GFT}
}
\btext{.10}{.12}{.30}{\Large ProofPower rules of inference}
\putfig{.45}{.25}{.50}{pp}





\normalpage{CS113 LAB 4 - Example 4.3}{
\begin{tabular}{|c|c||c|c|c|c|c|}
\hline
$ p $ & $ q $ & $ (p \rightarrow q) $ & $  \sim (p \rightarrow q) $ & $  \sim q $ & $ (p \wedge  \sim q) $ & $ ( \sim (p \rightarrow q) \leftrightarrow (p \wedge  \sim q)) $ \\
\hline
F & F & T & F & T & F & T \\
F & T & T & F & F & F & T \\
T & F & F & T & T & T & T \\
T & T & T & F & F & F & T \\
\hline
\end{tabular}

\begin{GFT}{SML}
\+fun premise1[p, q] = conditional[p, q];\\
\+fun premise2[p, q] = conditional[q, p];\\
\+fun conclusion[p, q] = disjunction[p, q];\\
\+fun v[p, q] = isValid([premise1,premise2], conclusion, p, q);\\
\+isTautology(map v (ttgen(2, (true, false))));\\
\end{GFT}
}
\btext{.80}{.15}{.08}{
\begin{tabular}{l}
$p \rightarrow q$\\
$q \rightarrow p$\\
\hline
$\therefore p \lor q$
\end{tabular}}
\putfig{.62}{.37}{.30}{ex3pre1.eps}
\putfig{.62}{.44}{.30}{ex3pre2.eps}
\putfig{.62}{.53}{.30}{ex3con.eps}
\putfig{.62}{.62}{.30}{ex3isvalid.eps}
\putfig{.62}{.71}{.30}{ex3istautology.eps}




\normalpage{CS113 LAB 4 - Example 4.4}{
\begin{tabular}{|c|c||c|c|c|}
\hline
$ p $ & $ q $ & $ (p \rightarrow q) $ & $ ((p \rightarrow q) \wedge p) $ & $ (((p \rightarrow q) \wedge p) \rightarrow q) $ \\
\hline
T & T & T & T & T \\
T & F & F & F & T \\
F & T & T & F & T \\
F & F & T & F & T \\
\hline
\end{tabular}

\begin{GFT}{SML}
\+fun premise1[p, q] = conditional[p, q];\\
\+fun premise2[p, q] = p;\\
\+fun conclusion[p, q] = q;\\
\+fun v[p, q] = isValid([premise1,premise2], conclusion, p, q);\\
\+isTautology(map v (ttgen(2, (true, false))));\\
\+"Proof Power"\\
\+val premise = asm\_rule;\\
\+val modus\_ponens = \PrLE{}\_elim;\\
\+val p1 = premise \PrKM{} p \PrLE{} q \PrKO{};\\
\+val p2 = premise \PrKM{} p:BOOL \PrKO{};\\
\+modus\_ponens p1 p2;\\
\end{GFT}
}
\btext{.50}{.15}{.075}{
\begin{tabular}{l}
$p \rightarrow q$\\
$q$\\
\hline
$\therefore q$
\end{tabular}
}
\btext{.37}{.40}{.18}{This argument is \blue \textbf{valid} \black (This is Modus Ponens)}
\putfig{.65}{.15}{.30}{ex4pre1}
\putfig{.65}{.22}{.30}{ex4pre2}
\putfig{.65}{.29}{.30}{ex4con}
\putfig{.65}{.36}{.30}{ex4isvalid}
\putfig{.65}{.43}{.30}{ex4istautology}
\btext{.65}{.60}{.15}{\Large Proof Power}
\putfig{.50}{.68}{.45}{ex4pp}




\normalpage{CS113 LAB 4 - Example 4.5}{
\begin{tabular}{|c|c||c|c|c|c|c|}
\hline
$ p $ & $ q $ & $ (p \rightarrow q) $ & $  \sim q $ & $  \sim p $ & $ ( \sim q \rightarrow  \sim p) $ & $ ((p \rightarrow q) \leftrightarrow ( \sim q \rightarrow  \sim p)) $ \\
\hline
F & F & T & T & T & T & T \\
F & T & T & F & T & T & T \\
T & F & F & T & F & F & T \\
T & T & T & F & F & T & T \\
\hline
\end{tabular}

\begin{GFT}{SML}
\+fun premise1[p, q] = conditional[p, q];\\
\+fun premise2[p, q] = not q;\\
\+fun conclusion[p, q] = not p;\\
\+fun v[p, q] = isValid([premise1, premise2], conclusion, p, q);\\
\+isTautology(map v (ttgen(2, (true, false))));\\
\end{GFT}
}
\btext{.65}{.15}{.075}{
\begin{tabular}{l}
$p \rightarrow q$\\
$\sim q$\\
\hline
$\therefore \sim p$
\end{tabular}
}
\btext{.05}{.67}{.50}{\Large This argument is \blue \textbf{valid} \black (This is Modus Tollens)}
\putfig{.65}{.36}{.30}{ex5pre1}
\putfig{.65}{.43}{.30}{ex5pre2}
\putfig{.65}{.50}{.30}{ex5con}
\putfig{.65}{.57}{.30}{ex5isvalid}
\putfig{.65}{.65}{.30}{ex5istautology}




\normalpage{CS113 LAB 4 - Problem 4.3}{
Problem 4.3\\
\begin{tabular}{|c|c|c||c|c|c|c|c|}
\hline
$ p $ & $ q $ & $ r $ & $  \sim p $ & $ ( \sim p \vee q) $ & $ (( \sim p \vee q) \rightarrow r) $ & $ ((( \sim p \vee q) \rightarrow r) \wedge ( \sim p \vee q)) $ & $ (((( \sim p \vee q) \rightarrow r) \wedge ( \sim p \vee q)) \rightarrow r) $ \\
\hline
T & T & T & F & T & T & T & T \\
T & T & F & F & T & F & F & T \\
T & F & T & F & F & T & F & T \\
T & F & F & F & F & T & F & T \\
F & T & T & T & T & T & T & T \\
F & T & F & T & T & F & F & T \\
F & F & T & T & T & T & T & T \\
F & F & F & T & T & F & F & T \\
\hline
\end{tabular}

\begin{GFT}{SML}
\+val premise = asm\_rule;\\
\+val modus\_ponens = \PrLE{}\_elim;\\
\+val p1 = premise \PrKM{} \PrLD{}p \PrLC{} q \PrLE{} r \PrKO{};\\
\+val p2 = premise \PrKM{} \PrLD{}p \PrLC{} q \PrKO{};\\
\+modus\_ponens p1 p2;\\
\end{GFT}
}
\btext{.35}{.55}{.12}{
\begin{tabular}{l}
$\sim p \lor q \rightarrow r$\\
$\sim p \lor q$\\
\hline
$\therefore r$
\end{tabular}
}
\btext{.35}{.80}{.18}{This argument is \blue \textbf{valid} \black (This is Modus Ponens)}
\putfig{.50}{.55}{.45}{pb3}




\normalpage{CS113 LAB 4 - Problem 4.4}{
\begin{tabular}{|c|c||c|c|c|}
\hline
$ p $ & $ q $ & $ (p \rightarrow q) $ & $ ((p \rightarrow q) \wedge q) $ & $ (((p \rightarrow q) \wedge q) \rightarrow p) $ \\
\hline
T & T & T & T & T \\
T & F & F & F & T \\
F & T & T & T & F \\
F & F & T & F & T \\
\hline
\end{tabular}

\begin{GFT}{SML}
\+fun premise1[p, q] = conditional[p, q];\\
\+fun premise2[p, q] = q;\\
\+fun conclusion[p, q] = p;\\
\+fun v[p, q] = isValid([premise1,premise2], conclusion, p, q);\\
\+isTautology(map v (ttgen(2, (true, false))));\\
\end{GFT}
}
\btext{.50}{.15}{.06}{
\begin{tabular}{l}
$p \rightarrow q$\\
$q$\\
\hline
$\therefore p$
\end{tabular}
}
\btext{.05}{.70}{.27}{\Large This argument is \red \textbf{invalid}}
\putfig{.55}{.36}{.40}{pb4pre1}
\putfig{.55}{.45}{.40}{pb4pre2}
\putfig{.55}{.54}{.40}{pb4con}
\putfig{.55}{.63}{.40}{pb4isvalid}
\putfig{.55}{.72}{.40}{pb4istautology}




\normalpage{CS113 LAB 4 - Problem 4.8}{
\begin{tabular}{|c|c||c|c|c|c|}
\hline
$ p $ & $ q $ & $ (p \vee q) $ & $  \sim p $ & $ ((p \vee q) \wedge  \sim p) $ & $ (((p \vee q) \wedge  \sim p) \rightarrow q) $ \\
\hline
T & T & T & F & F & T \\
T & F & T & F & F & T \\
F & T & T & T & T & T \\
F & F & F & T & F & T \\
\hline
\end{tabular}

\begin{GFT}{SML}
\+fun premise1[p, q] = disjunction[p, q];\\
\+fun premise2[p, q] = not p;\\
\+fun conclusion[p, q] = q;\\
\+fun v[p, q] = isValid([premise1, premise2], conclusion, p, q);\\
\+isTautology(map v (ttgen(2, (true, false))));\\
\end{GFT}
}
\btext{.55}{.15}{.06}{
\begin{tabular}{l}
$p \lor q$\\
$\sim p$\\
\hline
$\therefore q$
\end{tabular}
}
\btext{.05}{.70}{.25}{\Large This argument is \blue \textbf{valid}}
\putfig{.55}{.36}{.40}{pb8pre1}
\putfig{.55}{.45}{.40}{pb8pre2}
\putfig{.55}{.54}{.40}{pb8con}
\putfig{.55}{.63}{.40}{pb8isvalid}
\putfig{.55}{.72}{.40}{pb8istautology}




\normalpage{CS113 LAB 4 - Problem 4.13}{
\resizebox{\columnwidth}{!}{
\begin{tabular}{|c|c||c|c|c|c|c|c|c|}
\hline
$ p $ & $ q $ & $  \sim p $ & $ ( \sim p \rightarrow q) $ & $  \sim q $ & $ ( \sim q \rightarrow p) $ & $ (( \sim p \rightarrow q) \wedge ( \sim q \rightarrow p)) $ & $ ( \sim p \vee  \sim q) $ & $ ((( \sim p \rightarrow q) \wedge ( \sim q \rightarrow p)) \rightarrow ( \sim p \vee  \sim q)) $ \\
\hline
T & T & F & T & F & T & T & F & F \\
T & F & F & T & T & T & T & T & T \\
F & T & T & T & F & T & T & T & T \\
F & F & T & F & T & F & F & T & T \\
\hline
\end{tabular}
}
\\~\\
\textit{\textbf{Premise 1}}: If Tom is not on team A, then Hua is on team B.\\
\textit{\textbf{Premise 2}}: If Hua is not on team B, then Tom is on team A.\\
\textit{\textbf{Therefore}}: Tom is not on team A or Hua is not on team B.\\
\begin{GFT}{SML}
\+val p = "Tom is on team A";\\
\+val q = "Hua is on team B";\\
\+fun premise1[p, q] = conditional[not p, q];\\
\+fun premise2[p, q] = conditional[not q, p];\\
\+fun conclusion[p, q] = disjunction[not p, not q];\\
\+fun v[p, q] = isValid([premise1, premise2], conclusion, p, q);\\
\+isTautology(map v (ttgen(2, (true, false))));\\
\end{GFT}
}
\btext{.40}{.50}{.06}{
\begin{tabular}{l}
$p \lor q$\\
$\sim p$\\
\hline
$\therefore q$
\end{tabular}
}
\btext{.55}{.85}{.27}{\Large This argument is \red \textbf{invalid}}
\putfig{.55}{.36}{.40}{pb13pre1}
\putfig{.55}{.45}{.40}{pb13pre2}
\putfig{.55}{.54}{.40}{pb13con}
\putfig{.55}{.63}{.40}{pb13isvalid}
\putfig{.55}{.72}{.40}{pb13istautology}




\normalpage{CS113 LAB 4 - Problem 4.16}{
\\~\\
\begin{GFT}{SML}
\+val L1 = premise \PrKM{} \PrLD{}p \PrLC{} q \PrLE{} r \PrKO{};\\
\+val L2 = premise \PrKM{} s \PrLC{} \PrLD{}q \PrKO{};\\
\+val L3 = premise \PrKM{} \PrLD{}t \PrKO{};\\
\+val L4 = premise \PrKM{} p \PrLE{} t \PrKO{};\\
\+val L5 = premise \PrKM{} \PrLD{}p \PrLB{} r \PrLE{} \PrLD{}s \PrKO{};\\
\+val L6 = modus\_tollens L4 L3;\\
\+val L7 = disjunctive\_addition \PrKM{} q:BOOL \PrKO{} L6;\\
\+val L8 = modus\_ponens L1 L7;\\
\+val L9 = conjunctive\_addition L6 L8;\\
\+val L10 = modus\_ponens L5 L9;\\
\+val L11 = disjunctive\_syllogism L2 L10;\\
\end{GFT}
}
\btext{.05}{.15}{.60}{\Large Use Proofpower rules of inference to derive the conclusion.}
\btext{.45}{.25}{.50}{
\begin{tabular}{rll}
1. & $\sim p \lor q \rightarrow r$		& premise\\
2. & $s \lor \sim q$				& premise\\
3. & $\sim t$					& premise\\
4. & $p \rightarrow t$				& premise\\
5. & $\sim p \land r \rightarrow \sim s$	& premise\\
\hdashline
6. & $\sim p$					& modus tollens\\
7. & $\sim p \lor q$				& disjunctive addtion 6\\
8. & $r$					& modus ponens 1 7\\
9. & $\sim p \land r$				& conjunctive addition 6 8\\
10.& $\sim s$					& modus ponens 5 9\\
11.& $\sim q$					& disjunctive syllogism 2 10\\
\hline
   & $\therefore \quad \sim q$			& conclusion
\end{tabular}
}




\normalpage{CS113 LAB 4 - Problem 4.16}{}
\putfig{.05}{.20}{.90}{pb16}




\normalpage{CS113 LAB 4 - Problem 4.17}{
\\~\\
\begin{GFT}{SML}
\+val L1 = premise \PrKM{} \PrLD{}p \PrLE{} r \PrLB{} \PrLD{}s \PrKO{};\\
\+val L2 = premise \PrKM{} t \PrLE{} s \PrKO{};\\
\+val L3 = premise \PrKM{} u \PrLE{} \PrLD{}p \PrKO{};\\
\+val L4 = premise \PrKM{} \PrLD{}w \PrKO{};\\
\+val L5 = premise \PrKM{} u \PrLC{} w \PrKO{};\\
\+val L6 = disjunctive\_syllogism L5 L4;\\
\+val L7 = modus\_ponens L3 L6;\\
\+val L8 = modus\_ponens L1 L7;\\
\+val L9 = conjunctive\_simplification L8;\\
\+val L10 = modus\_tollens L2 L9;\\
\+val L11 = disjunctive\_addition \PrKM{} w:BOOL \PrKO{} L10;\\
\end{GFT}
}
\btext{.05}{.15}{.80}{\large Use the valid argument forms of this section to deduce the conclusion from the premises.}
\btext{.45}{.25}{.50}{
\begin{tabular}{rll}
1. & $\sim p \rightarrow r \land \sim s$	& premise\\
2. & $t \rightarrow s$				& premise\\
3. & $u \rightarrow \sim p$			& premise\\
4. & $\sim w$					& premise\\
5. & $u \lor w$					& premise\\
\hdashline
6. & $u$					& disjunctive syllogism\\
7. & $\sim p$					& modus ponens 3 6\\
8. & $r \land \sim s$				& modus ponens 1 7\\
9. & $\sim s$					& conjunctive simplification L8\\
10.& $\sim t$					& modus tollens L2 L9\\
11.& $\sim t \lor w$				& disjunctive addition L10\\
\hline
   & $\therefore \quad \sim t \lor w$		& conclusion
\end{tabular}
}




\normalpage{CS113 LAB 4 - Problem 4.17}{}
\putfig{.05}{.20}{.90}{pb17}




\normalpage{CS113 LAB 4 - Problem 4.18}{
\\~\\
\begin{GFT}{SML}
\+val L1 = premise \PrKM{} \PrLD{}(p \PrLC{} q) \PrLE{} r \PrKO{};\\
\+val L2 = premise \PrKM{} \PrLD{}p \PrKO{};\\
\+val L3 = premise \PrKM{} \PrLD{}r \PrKO{};\\
\+val L4 = modus\_tollens L1 L3;\\
\+val L5 = double\_negation L4;\\
\+val L6 = disjunctive\_syllogism L5 L2;\\
\end{GFT}
}
\btext{.05}{.15}{.90}{\Large Use the valid argument forms of this section to deduce the conclusion from the premises.}
\btext{.05}{.65}{.90}{\Large Notice that ProofPower doesn't apply double-negation automatically,\\ so inserted double-negation manually. (\magenta L5\black )}
\btext{.45}{.25}{.50}{\begin{tabular}{rll}
1. & $\sim p \lor q \rightarrow r$		& premise\\
2. & $\sim p$					& premise\\
3. & $\sim r$					& premise\\
\hdashline
4. & $\sim \sim(p \lor q)$			& modus tollens 1 3\\
\magenta 5. & $p \lor q$			& double-negation 4\\ \black
6. & $q$					& disjunctive syllogism 2 5\\
\hline
   & $\therefore \quad \sim s$			& conclusion
\end{tabular}}




\normalpage{CS113 LAB 4 - Problem 4.18}{}
\putfig{.05}{.20}{.90}{pb18}




\end{document}
