\documentclass{ffslides}
\ffpage{60}{\numexpr 9/16}
\usepackage{fancyvrb}
\begin{document}
\fvset{numbers=left,numbersep=3pt,frame=single,firstnumber=1,tabsize=4}


\normalpage{Lab 3 - lab.h}{
	\VerbatimInput[firstline=18,lastline=53]{../includes/lab.h}
}
\btext{.30}{.20}{.30}{\large Libraries:\\
iostream: for for stdin/stdout via terminal\\
fstream: for read/write file\\
vector: for vector\\
abc: ABC to SOX associative table}
\btext{.30}{.35}{.30}{\large Structs:\\
Note: for the each note\\
Fragment: for the stack element\\
MyStack: stack\\~\\
\large Global:\\
TEMPO: for the tempo of the song}


\normalpage{Lab 3 - lab.h}{
	\VerbatimInput[firstline=56,firstnumber=37]{../includes/lab.h}
}
\btext{.10}{.10}{.30}{\large Function Declaration\\
}


\normalpage{Lab 3 - abc.h}{
	\VerbatimInput[firstline=18,lastline=59]{../includes/abc.h}
	\VerbatimInput[firstline=18,lastline=40]{../srcs/abc.cpp}
}
\btext{.40}{.25}{.40}{\large ABC header:\\
This is a custom header and cpp file for\\
translating ABC notation to SOX notation.\\
There is an associative table that\\
converts ABC notation to SOX notation.}
\btext{.40}{.60}{.40}{\large ABC:\\
The associative table is declared here\\
(only for measure notations), and\\
it will be exported as a global variable.}


\normalpage{Lab 3 - readABCfile.cpp}{
	\VerbatimInput[firstline=18]{../srcs/readABCfile.cpp}
}
\btext{.45}{.30}{.40}{\large readABCfile:\\
This function opens the file and reads the data.\\
Then, store the data into a vector of strings.\\
It will skip the information part in the beginning,\\
only store L:, which is the tempo.}
\btext{.45}{.75}{.40}{\large getTempo:\\
This function calculates the tempo of the song\\
based on 1/4 is tempo 1.}

\normalpage{Lab 3 - convertABCtoSOX.cpp}{
	\VerbatimInput[firstline=18,lastline=59]{../srcs/convertABCtoSOX.cpp}
}
\btext{.55}{.30}{.35}{\large convertABCtoSOX:\\
This function checks the ABC notation and\\
covert it into the SOX notation.\\
Also, it gets the duration and fragments.\\~\\
It checks every letter in the string and\\
gets the note.\\
For the duration, gets the number using atoi.\\
At the end it changes lower case to upper case\\
using AND bit operator.}


\normalpage{Lab 3 - createPlayCmd.cpp}{
	\VerbatimInput[firstline=18,lastline=50]{../srcs/createPlayCmd.cpp}
}
\btext{.10}{.35}{.40}{\large createPlayCmd:\\
This function takes a Note struct and\\
creates a SOX command. It combines the command\\
template and the data in the struct.\\}



\normalpage{Lab 3 - playSong.cpp}{
	\VerbatimInput[firstline=18,lastline=59]{../srcs/playSong.cpp}
}
\btext{.53}{.30}{.40}{\large playSong:\\
This function takes the soxVector and plays the song.\\
First, it creates a new stack to store the indices.\\
If it meets repeat fragment while it plays the song,\\
it will push the current index into the stack.\\
Then, it plays the repeat part again and comes back to\\
the stored index using pop.\\
When the song ends, it destroy the stack.}


\normalpage{Lab 3 - playSong.cpp}{
	\VerbatimInput[firstline=60,firstnumber=43]{../srcs/playSong.cpp}
}


\normalpage{Lab 3 - myCreate.cpp}{
	\VerbatimInput[firstline=18,lastline=59]{../srcs/myCreate.cpp}
}
\btext{.45}{.25}{.40}{\large myCreate:\\
It creates a new stack.\\
If allocation fails, returns false.\\~\\
UNIT\_TEST:\\
I created several times with different sizes.\\
Then I checked if it returns true.\\
I couldn't test failure since I don't know\\
how to make 'new' fail.}


\normalpage{Lab 3 - myPush.cpp}{
	\VerbatimInput[firstline=18,lastline=59]{../srcs/myPush.cpp}
}
\btext{.45}{.25}{.40}{\large myPush:\\
It pushes the item into the stack.\\
If the stack is full, returns false.\\~\\
UNIT\_TEST:\\
I created a new stack with size 2.\\
Then, I push something twice and check if\\
the stack has the items in the proper order.\\
Also, I pushed one more time so that\\
I can check the failure when it is full.}

\normalpage{Lab 3 - myPop.cpp}{
	\VerbatimInput[firstline=18,lastline=59]{../srcs/myPop.cpp}
}
\btext{.45}{.25}{.40}{\large myPop:\\
It pops the item in the top of the stack.\\
If the stack is empty, returns false.\\~\\
UNIT\_TEST:\\
I pushed 2 items and check if it pops\\
in the proper order.\\
After that, I tried to pop the empty stack\\
and checked if it returns false.}


\normalpage{Lab 3 - myDestory.cpp}{
	\VerbatimInput[firstline=18,lastline=59]{../srcs/myDestroy.cpp}
}
\btext{.45}{.25}{.40}{\large myDestroy:\\
It deletes the stack and set it to null pointer.\\~\\
UNIT\_TEST:\\
Since delete functions doesn't return any bool\\
expression and the data stays even it was deleted,\\
it is hard to check success/fail of the function.\\
So, I guess it worked if the program doesn't\\
throw an execption error.}
\btext{.05}{.55}{.30}{\Huge \textsc{CATCH TEST}}
\putfig{.05}{.60}{.80}{catchtest.eps}


\normalpage{Lab 3 - Result Screenshots}{
}
\btext{.05}{.11}{.15}{\Huge \textsc{Convert}}
\putfig{.05}{.15}{.60}{convert.eps}
\putfig{.65}{.10}{.30}{convert_res.eps}


\normalpage{Lab 3 - Result Screenshots}{
}
\btext{.05}{.11}{.15}{\Huge \textsc{Result}}
\putfig{.05}{.15}{.80}{running.eps}


\end{document}
